\chapter{Modelo Proposto}
\label{ch:modeloProposto}

O modelo proposto neste \gls{tfm} consiste no desenvolvimento de um sistema ciberfísico para o controlo autónomo de embarcações do tipo \gls{usv}, concebido de forma modular e expansível. A modularidade é garantida pela utilização do barramento \gls{i2c} de comunicação entre os diferentes módulos de \emph{hardware}. A escolha do \gls{i2c} como base do sistema justifica-se pela sua simplicidade de implementação, baixo número de ligações necessárias e pela possibilidade de integrar múltiplos dispositivos no mesmo barramento através da atribuição de endereços únicos. Esta abordagem garante escalabilidade, permitindo a integração de novos sensores, atuadores, ou até mesmo diferentes microcontroladores, desde que estes suportem \gls{i2c}, sem alterações profundas na arquitetura.

De forma a apresentar esta proposta de forma estruturada, este capítulo encontra-se organizado em duas secções. Na Secção~\ref{sec:requisitos} são descritos os requisitos funcionais e não funcionais que orientam o desenvolvimento do sistema, identificando as funcionalidades essenciais e as propriedades globais de desempenho e escalabilidade. Na Secção~\ref{sec:abordagem} é detalhada a abordagem modular seguida, apresentando os principais módulos de \emph{hardware} e \emph{software} que compõem o sistema, bem como as suas interações. Esta estrutura permite compreender de forma clara tanto as bases conceptuais como a implementação prática do modelo proposto.

\section{Requisitos}
\label{sec:requisitos}

Os Requisitos Funcionais correspondem a descrições formais e detalhadas das funcionalidades que um sistema deve disponibilizar, especificando as ações que este deve ser capaz de executar e a forma como deve responder às diferentes interações do utilizador ou de outros sistemas.  

Por sua vez, os Requisitos Não Funcionais referem-se a propriedades globais do sistema que não estão diretamente associadas a uma funcionalidade específica, mas que condicionam o seu comportamento e qualidade de operação. Entre estes incluem-se aspetos como o desempenho, a segurança, a usabilidade, a escalabilidade e a fiabilidade, que são determinantes para garantir a robustez e adequação da solução em cenários reais de utilização.

Os requisitos principais definidos para o sistema proposto neste \gls{tfm} são:

\begin{enumerate}
    \item O sistema deve utilizar comunicação de longo alcance baseada em tecnologia \gls{lora}, de forma a permitir telemetria e atualização de rotas em tempo real, mesmo em cenários remotos.  
    Requisito funcional, dado que descreve explicitamente uma ação que o sistema deve executar, isto é, a capacidade de estabelecer comunicação sem fios de longo alcance para troca de dados críticos com a estação de controlo.

    \item O sistema deve ser expansível através do barramento \gls{i2c}, assegurando a fácil integração de novos módulos.  
    Requisito não funcional, uma vez que não descreve uma funcionalidade direta do sistema, mas sim uma característica de extensibilidade e escalabilidade, que garante a adaptabilidade futura da arquitetura.  
\end{enumerate}

Desta forma, observa-se que o primeiro requisito define uma funcionalidade central para a operação do veículo, indispensável para o cumprimento dos objetivos de navegação autónoma e monitorização remota enquanto o segundo requisito reflete uma propriedade estrutural do sistema, relacionada com a sua capacidade de evolução e integração modular.

\section{Abordagem}
\label{sec:abordagem}

Para assegurar a operação autónoma, o sistema integra diferentes módulos, tal como apresentado na Figura \ref{fig:modulos-usv}. 

\begin{figure}[H]
  \centering
  \resizebox{0.8\textwidth}{!}{%
    % generated by Plantuml 1.2024.3       
\definecolor{plantucolor0000}{RGB}{241,241,241}
\definecolor{plantucolor0001}{RGB}{24,24,24}
\definecolor{plantucolor0002}{RGB}{0,0,0}
\definecolor{plantucolor0003}{RGB}{255,255,224}
\definecolor{plantucolor0004}{RGB}{173,216,230}
\begin{tikzpicture}[yscale=-1
,pstyle0/.style={color=plantucolor0001,fill=plantucolor0000,line width=0.5pt}
,pstyle2/.style={color=plantucolor0001,fill=plantucolor0004,line width=0.5pt}
,pstyle4/.style={color=plantucolor0001,line width=1.0pt}
,pstyle5/.style={color=plantucolor0001,fill=plantucolor0001,line width=1.0pt}
,pstyle6/.style={color=plantucolor0001,line width=1.0pt,dash pattern=on 7.0pt off 7.0pt}
]
\draw[pstyle0] (150pt,12pt) arc (180:270:5pt) -- (155pt,7pt) -- (302.6571pt,7pt) arc (270:360:5pt) -- (307.6571pt,12pt) -- (307.6571pt,41.0679pt) arc (0:90:5pt) -- (302.6571pt,46.0679pt) -- (155pt,46.0679pt) arc (90:180:5pt) -- (150pt,41.0679pt) -- cycle;
\node at (160pt,17pt)[below right,color=black]{Microcontrolador};
\draw[color=plantucolor0001,fill=plantucolor0003,line width=0.5pt] (7pt,112pt) arc (180:270:5pt) -- (12pt,107pt) -- (85.9636pt,107pt) arc (270:360:5pt) -- (90.9636pt,112pt) -- (90.9636pt,160.1358pt) arc (0:90:5pt) -- (85.9636pt,165.1358pt) -- (12pt,165.1358pt) arc (90:180:5pt) -- (7pt,160.1358pt) -- cycle;
\node at (17pt,117pt)[below right,color=black]{\textit{\guillemotleft comm\guillemotright }};
\node at (29.0524pt,136.0679pt)[below right,color=black]{LoRa};
\draw[pstyle2] (126pt,112pt) arc (180:270:5pt) -- (131pt,107pt) -- (207.4966pt,107pt) arc (270:360:5pt) -- (212.4966pt,112pt) -- (212.4966pt,160.1358pt) arc (0:90:5pt) -- (207.4966pt,165.1358pt) -- (131pt,165.1358pt) arc (90:180:5pt) -- (126pt,160.1358pt) -- cycle;
\node at (136pt,117pt)[below right,color=black]{\textit{\guillemotleft sensor\guillemotright }};
\node at (154.2637pt,136.0679pt)[below right,color=black]{GPS};
\draw[pstyle2] (247pt,112pt) arc (180:270:5pt) -- (252pt,107pt) -- (328.4966pt,107pt) arc (270:360:5pt) -- (333.4966pt,112pt) -- (333.4966pt,160.1358pt) arc (0:90:5pt) -- (328.4966pt,165.1358pt) -- (252pt,165.1358pt) arc (90:180:5pt) -- (247pt,160.1358pt) -- cycle;
\node at (257pt,117pt)[below right,color=black]{\textit{\guillemotleft sensor\guillemotright }};
\node at (274.1626pt,136.0679pt)[below right,color=black]{IMU};
\draw[pstyle0] (368pt,121.5pt) arc (180:270:5pt) -- (373pt,116.5pt) -- (425.0889pt,116.5pt) arc (270:360:5pt) -- (430.0889pt,121.5pt) -- (430.0889pt,150.5679pt) arc (0:90:5pt) -- (425.0889pt,155.5679pt) -- (373pt,155.5679pt) arc (90:180:5pt) -- (368pt,150.5679pt) -- cycle;
\node at (378pt,126.5pt)[below right,color=black]{PWM};
\draw[pstyle0] (290.5pt,269pt) arc (180:270:5pt) -- (295.5pt,264pt) -- (366.9868pt,264pt) arc (270:360:5pt) -- (371.9868pt,269pt) -- (371.9868pt,298.0679pt) arc (0:90:5pt) -- (366.9868pt,303.0679pt) -- (295.5pt,303.0679pt) arc (90:180:5pt) -- (290.5pt,298.0679pt) -- cycle;
\node at (300.5pt,274pt)[below right,color=black]{Motor 1};
\draw[pstyle0] (406.5pt,269pt) arc (180:270:5pt) -- (411.5pt,264pt) -- (486.7pt,264pt) arc (270:360:5pt) -- (491.7pt,269pt) -- (491.7pt,298.0679pt) arc (0:90:5pt) -- (486.7pt,303.0679pt) -- (411.5pt,303.0679pt) arc (90:180:5pt) -- (406.5pt,298.0679pt) -- cycle;
\node at (416.5pt,274pt)[below right,color=black]{Motor N};
\draw[pstyle0] (48.8333pt,252.5pt) ellipse (8pt and 8pt);
\draw[color=plantucolor0001,line width=0.5pt] (48.8333pt,260.5pt) -- (48.8333pt,287.5pt)(35.8333pt,268.5pt) -- (61.8333pt,268.5pt)(48.8333pt,287.5pt) -- (35.8333pt,302.5pt)(48.8333pt,287.5pt) -- (61.8333pt,302.5pt);
\node at (31.5pt,304pt)[below right,color=black]{User};
\draw[pstyle4] (218.55pt,46.23pt) ..controls (209.21pt,62.95pt) and (195.41pt,87.68pt) .. (184.74pt,106.8pt);
\draw[pstyle4] (239.63pt,46.23pt) ..controls (249.11pt,62.95pt) and (263.15pt,87.68pt) .. (274pt,106.8pt);
\draw[pstyle4] (258.61pt,46.23pt) ..controls (289.81pt,65.96pt) and (338.65pt,96.84pt) .. (369.73pt,116.49pt);
\draw[pstyle4] (197.65pt,46.23pt) ..controls (168.13pt,63.85pt) and (123.72pt,90.38pt) .. (91.12pt,109.84pt);
\draw[pstyle4] (390.25pt,155.72pt) ..controls (377.19pt,183.66pt) and (355.2889pt,230.5136pt) .. (342.2589pt,258.4136pt);
\draw[pstyle5] (339.72pt,263.85pt) -- (347.1526pt,257.3881pt) -- (341.8358pt,259.3197pt) -- (339.9041pt,254.0029pt) -- (339.72pt,263.85pt) -- cycle;
\draw[pstyle4] (405.43pt,155.72pt) ..controls (415.03pt,183.66pt) and (431.0496pt,230.2758pt) .. (440.6396pt,258.1758pt);
\draw[pstyle5] (442.59pt,263.85pt) -- (443.4472pt,254.0385pt) -- (440.9647pt,259.1215pt) -- (435.8817pt,256.639pt) -- (442.59pt,263.85pt) -- cycle;
\draw[pstyle6] (371.56pt,283.5pt) ..controls (383.16pt,283.5pt) and (394.75pt,283.5pt) .. (406.35pt,283.5pt);
\draw[pstyle6] (49pt,165.05pt) ..controls (49pt,187.43pt) and (49pt,213.02pt) .. (49pt,237.81pt);
\draw[pstyle5] (49pt,243.81pt) -- (53pt,234.81pt) -- (49pt,238.81pt) -- (45pt,234.81pt) -- (49pt,243.81pt) -- cycle;
\node at (50pt,196pt)[below right,color=black]{LoRa};
\end{tikzpicture}
%
  }
  \caption{Diagrama da arquitetura modular do \gls{usv}}}
  \label{fig:modulos-usv}
\end{figure}

O \acrfull{gps} é responsável pela determinação da posição geográfica da embarcação, fornecendo coordenadas de latitude e longitude necessárias para o seguimento de rotas. O \acrfull{imu}, por sua vez, mede aceleração, rotação e orientação da embarcação nos eixos tridimensionais, permitindo a correção de trajetória e a compensação de efeitos de correntes e ondas. A interface de comunicação \acrfull{lora} é utilizada para o envio de dados de telemetria, a receção de novas rotas e a indicação do estado de autonomia (manual ou automático). No modo manual, possibilita ainda a transmissão de comandos de movimento (frente, trás, esquerda e direita), garantindo uma comunicação fiável em cenários de operação prolongada e em zonas sem infraestrutura de telecomunicações.

% Adicionalmente, foi integrado um módulo de armazenamento em cartão SD, cuja função é registar localmente os dados de telemetria recolhidos durante a operação. Este armazenamento persistente assegura que a informação relativa à trajetória, parâmetros de navegação e estados do sistema possa ser posteriormente analisada, mesmo em situações em que a comunicação de longo alcance esteja temporariamente indisponível. Esta funcionalidade é essencial para a validação experimental, análise pós-missão e suporte a algoritmos de diagnóstico e melhoria contínua.

O sistema deverá ainda ser capaz de controlar até quatro motores de forma independente, assim como no robô didático \cite{didactic-robot-thesis}, recorrendo a sinais \gls{pwm} gerados pelo microcontrolador. O controlo distribuído de motores oferece flexibilidade na navegação e aumenta a manobrabilidade do \gls{usv}.

A abordagem modular proposta assegura que a arquitetura não se limita aos sensores e interfaces inicialmente integrados, permitindo a evolução futura do sistema com a adição de novos módulos, como sensores de ambiente, câmaras ou interfaces de comunicação alternativas. Desta forma, o modelo apresentado constitui uma solução escalável, robusta e alinhada com os requisitos de flexibilidade e adaptabilidade para missões em ambientes marítimos complexos.
