% !TeX spellcheck = pt_PT2

\chapter{Resumo}

Este \acrfull{tfm} descreve o desenvolvimento de um sistema ciberfísico para controlo autónomo de embarcações do tipo \acrfull{usv}. O sistema foi concebido de forma modular, expansível e orientado para a integração de múltiplos motores, sensores e interfaces de comunicação, assegurando robustez, baixo custo e adaptabilidade a diferentes cenários operacionais. Todo o código desenvolvido encontra-se disponível em repositório \emph{open source}, promovendo transparência, colaboração e reutilização científica.

A arquitetura proposta baseia-se na utilização de um microcontrolador \acrfull{esp32} com suporte nativo a \acrfull{i2c}, \acrfull{uart} e \acrfull{lora}, para controlo dos diferentes módulos, incluindo sensores e atuadores. Entre os principais módulos incluem-se: i) propulsores \emph{brushless} controlados por \acrfull{esc}, capazes de fornecer empuxo bidirecional; ii) um \acrfull{imu}, que fornece medições de guinada (\emph{yaw}), arfagem (\emph{pitch}) e rotação (\emph{roll}) para garantir a estabilidade e a correção de trajetória; iii) um recetor \acrfull{gps}, responsável pela navegação precisa através de coordenadas geoespaciais; e iv) um módulo de comunicação \acrfull{lora}, que assegura telemetria de longo alcance e receção de rotas em tempo real com baixo consumo energético.

Do ponto de vista de \emph{software}, a solução foi estruturada em módulos independentes que encapsulam funções de propulsão, sensorização, comunicação e controlo. Esta abordagem promove a reutilização de código, facilita a manutenção e garante a escalabilidade do sistema. A comunicação foi otimizada através do uso de \emph{Protocol Buffers} (\texttt{Protobuf}), reduzindo significativamente o tempo de transmissão (\acrfull{toa}) em mensagens \acrfull{lora}, o que contribui para maior eficiência energética, menor probabilidade de colisões e aumento do alcance efetivo.

A validação experimental incluiu testes incrementais de \emph{hardware} e \emph{software}, desde o controlo isolado de propulsores até à integração de todos os módulos num protótipo funcional. Os resultados práticos, obtidos em ambiente real no \acrfull{rex25}, demonstraram a capacidade do sistema em seguir rotas definidas por \emph{waypoints} \acrfull{gps}, armazenar dados de telemetria e operar em modo manual ou automático. Este processo de validação operacional, em colaboração com a Marinha Portuguesa, reforça a aplicabilidade do sistema em cenários próximos de missão real.

Conclui-se que o sistema desenvolvido constitui uma base sólida para futuras evoluções em \acrfull{usv}, nomeadamente a aplicação de controladores avançados (\acrfull{pid} ou \acrfull{lqr}), a integração em plataformas marítimas de maior escala, e a utilização em cenários reais de monitorização ambiental, investigação científica e operações costeiras. Perspetiva-se ainda a exploração de técnicas de inteligência artificial e aprendizagem automática para otimização da navegação e da tomada de decisão em tempo real. O contexto português, em particular a sua vasta \acrfull{zee}, constitui um campo de aplicação natural para este tipo de soluções, contribuindo para o desenvolvimento científico e tecnológico nacional.

\palavraschave{\acrshort{usv}; Sistemas Ciberfísicos; \acrshort{lora}; \acrshort{esp32}; Navegação Autónoma; \texttt{Protobuf}.}
