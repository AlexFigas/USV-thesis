\newpage
\vspace*{\fill}
\begin{flushright}
\textit{"O cientista não procura um resultado imediato. Ele não espera que as suas ideias avançadas sejam prontamente aceites. O seu trabalho é lançar a base para aqueles que virão."} \\
--- Nikola Tesla
\end{flushright}
\vspace*{\fill}

\chapter{Agradecimentos}

Em primeiro lugar, quero expressar a minha profunda gratidão ao \acrfull{isel}, instituição que foi a minha casa ao longo de cinco anos (Licenciatura e Mestrado). Aqui encontrei não só os maiores desafios académicos da minha vida, mas também oportunidades únicas de crescimento pessoal e profissional. Foram muitas noites mal dormidas, momentos de frustração e superação, mas também conquistas que culminaram na realização deste \acrfull{tfm}.  

Agradeço igualmente à \acrfull{enidh}, que nos últimos meses se tornou a minha segunda casa, disponibilizando não apenas as condições necessárias para o desenvolvimento deste trabalho, mas também a embarcação e os espaços para ensaios. Graças a esta colaboração foi possível participar no exercício \gls{rex25}, um subevento do \gls{repmus25}, validando o protótipo em ambiente real.  

Quero também agradecer à Marinha Portuguesa e à Escola Naval pela oportunidade de integrar o \gls{repmus25} e o \gls{rex25}, onde este trabalho pôde ser colocado à prova em cenários operacionais exigentes. Esta experiência representou um marco na validação prática do sistema e um reconhecimento da importância de iniciativas académicas em contextos reais.  

Um agradecimento muito especial ao Professor Carlos Gonçalves, pela orientação incansável ao longo da licenciatura e do mestrado. O seu rigor, dedicação e espírito de exigência foram fundamentais para a conclusão deste percurso. Recordo as inúmeras noites no \gls{isel}, que se prolongavam até tarde, e que foram decisivas para transformar ideias em resultados concretos.  

Agradeço igualmente ao Professor Pedro Teodoro, pelo apoio na compreensão de conceitos ligados às embarcações e pela disponibilização da infraestrutura e embarcação necessárias para a realização dos testes.  

À minha família, em particular aos meus pais, agradeço por estarem sempre presentes e pelo apoio incondicional em todos os momentos. Sem o vosso incentivo e confiança, nada disto teria sido possível.  

Aos colegas e amigos que me acompanharam ao longo desta jornada académica, deixo também um sincero agradecimento. Foram muitas as vezes em que tive de recusar convites ou abdicar de momentos convosco devido às exigências deste mestrado, mas o vosso apoio e compreensão foram fundamentais para chegar até aqui.  

Um agradecimento também à minha empresa, Critical Techworks, pela flexibilidade e pelo apoio concedido durante os dois anos de mestrado. A possibilidade de conciliar a vida profissional com os estudos foi crucial para tornar este percurso viável.  

Finalmente, agradeço a mim próprio. Conciliar um mestrado com uma carreira profissional como trabalhador-estudante foi uma tarefa exigente e, por vezes, desgastante. No entanto, este trabalho é também prova de que, com esforço, dedicação e resiliência, é possível alcançar qualquer objetivo. Agradeço a mim por acreditar, por não desistir e por ser fiel a este caminho até ao fim.

\vspace*{\fill}
\begin{flushright}
\textit{"Conseguimos! Conseguimos, Portugal, Lisboa! Esperávamos, desejávamos, conseguimos! Vitória!"} \\
--- Doutor Marcelo Rebelo de Sousa
\end{flushright}
\vspace*{\fill}