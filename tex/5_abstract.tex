% !TeX spellcheck = en_US

\chapter{Abstract}

This \acrfull{tfm} describes the development of a cyber-physical system for the autonomous control of \acrfull{usv}. The system was designed to be modular, extensible, and oriented towards the integration of multiple motors, sensors, and communication interfaces, ensuring robustness, low cost, and adaptability to different operational scenarios. All developed code is available in an open source repository, promoting transparency, collaboration, and scientific reuse.

The proposed architecture is based on an \acrfull{esp32} microcontroller with native support for \acrfull{i2c}, \acrfull{uart}, and \acrfull{lora}, responsible for controlling the different modules, including sensors and actuators. The main modules include: i) brushless thrusters controlled by an \acrfull{esc}, capable of providing bidirectional thrust; ii) an \acrfull{imu}, supplying yaw, pitch, and roll measurements to ensure stability and trajectory correction; iii) a \acrfull{gps} receiver, responsible for accurate navigation through geospatial coordinates; and iv) a \acrfull{lora} communication module, enabling long-range telemetry and real-time route updates with low energy consumption.

From a software perspective, the solution was structured into independent modules that encapsulate propulsion, sensing, communication, and control functions. This approach fosters code reusability, simplifies maintenance, and ensures system scalability. Communication efficiency was enhanced through the use of Protocol Buffers (Protobuf), which significantly reduced the transmission time (\acrfull{toa}) of \acrfull{lora} messages, contributing to higher energy efficiency, lower collision probability, and increased effective range.

Experimental validation included incremental hardware and software tests, ranging from isolated thruster control to the full integration of all modules into a functional prototype. Practical results, obtained in a real-world environment during \acrfull{rex25}, demonstrated the system's ability to follow \acrfull{gps} waypoint-defined routes, store telemetry data, and operate in both manual and autonomous modes. This operational validation, conducted in collaboration with the Portuguese Navy, reinforced the applicability of the system in mission-like scenarios.

It is concluded that the developed system provides a solid foundation for future advancements in \acrfull{usv}, namely the application of advanced controllers (\acrfull{pid} or \acrfull{lqr}), integration into larger-scale maritime platforms, and use in real-world scenarios such as environmental monitoring, scientific research, and coastal operations. Furthermore, the exploration of artificial intelligence and machine learning techniques is foreseen to optimize navigation and real-time decision-making. The Portuguese context, particularly its vast Exclusive Economic Zone (EEZ), represents a natural field of application for this type of solution, contributing to national scientific and technological development.

\keywords{\acrshort{usv}; Cyber-Physical Systems; \acrshort{lora}; \acrshort{esp32}; Autonomous Navigation; Protobuf.}
