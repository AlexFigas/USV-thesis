\chapter{Conclusões e Trabalho Futuro}
\label{ch:conclusoesTrabalhoFuturo}

Nesta secção são apresentadas as conclusões do trabalho desenvolvido, bem como as linhas de evolução identificadas para o futuro. A Secção~\ref{sec:conclusoes} sumariza os principais resultados alcançados, destacando a relevância e o impacto do sistema proposto. A Secção~\ref{sec:trabalho-futuro} descreve as perspetivas de evolução do projeto, apontando para áreas de melhoria e novas funcionalidades que poderão ser exploradas em trabalhos futuros.

\section{Conclusões}
\label{sec:conclusoes}

O trabalho desenvolvido permitiu conceber e validar um sistema ciberfísico modular para o controlo autónomo de embarcações do tipo \gls{usv}. O protótipo integrou propulsores \emph{brushless}, sensores de navegação (\gls{gps} e \gls{imu}), comunicação de longo alcance via \gls{lora}, bem como uma arquitetura de \emph{software} estruturada em módulos reutilizáveis.  

A \gls{pcb} desenvolvida foi testada inicialmente num robô didático, comprovando a compatibilidade com o software existente e assegurando o correto funcionamento dos módulos de controlo e comunicação antes da sua integração no protótipo do \gls{usv}.  

O sistema foi posteriormente validado no âmbito do exercício internacional \gls{repmus25}, mais concretamente no subevento \gls{rex25}, onde demonstrou ser capaz de operar em modo manual e automático, executar rotas definidas por \emph{waypoints} \gls{gps} e manter comunicação fiável em tempo real com uma estação remota. Estes resultados comprovam a eficácia da arquitetura proposta e validam os princípios de modularidade, reutilização e escalabilidade que orientaram o desenvolvimento.  

De forma global, este trabalho representa um contributo relevante para a área dos sistemas ciberfísicos aplicados à robótica marítima, ao demonstrar que é possível implementar uma solução de baixo custo, reprodutível e extensível, capaz de servir como plataforma de investigação e desenvolvimento tecnológico.

\section{Trabalho Futuro}
\label{sec:trabalho-futuro}

Com base nos resultados alcançados, foram identificadas várias linhas de evolução que poderão potenciar o sistema desenvolvido e alargar o seu âmbito de aplicação.  

Em primeiro lugar, destaca-se a possibilidade de integração do sistema no \gls{usv}-enautica1, da Escola Superior Náutica Infante D. Henrique, o que permitirá validar o desempenho em cenários de maior escala e com exigências operacionais superiores.  

Outra vertente importante consiste na implementação de um sistema de controlo remoto manual, quer através de um comando dedicado semelhante aos utilizados em rádio controlo, quer através da integração com o comando desenvolvido em \cite{catamara-telecomandado}. Esta abordagem asseguraria a compatibilidade com plataformas operacionais já existentes e aumentaria a flexibilidade do controlo.  

A expansão do sistema para incluir o armazenamento de dados de telemetria num cartão SD é igualmente uma direção promissora, permitindo registar e analisar informações de desempenho de forma contínua.  

Do ponto de vista do controlo, a implementação e avaliação de estratégias mais avançadas, como controladores \gls{pid}, \gls{lqr}, de controlo ótimo ou baseados em aprendizagem por reforço, permitirá melhorar a precisão da trajetória e a robustez da navegação em ambientes dinâmicos.  

Outra linha de evolução consiste na exploração da navegação cooperativa entre múltiplos \gls{usv}, o que exigirá mecanismos de coordenação distribuída e comunicação em rede, seja através de \gls{lora} ou de tecnologias alternativas, como Wi-Fi de longo alcance ou comunicações mesh.  

A integração de sensores adicionais, tais como GNSS multiconstelação, câmaras ou radar de baixo custo, poderá reforçar a precisão da navegação e a capacidade de adaptação do sistema a ambientes marítimos mais complexos.  

A sustentabilidade energética constitui igualmente uma vertente de interesse, podendo ser melhorada através da utilização de painéis solares ou de sistemas híbridos de alimentação, de modo a aumentar a autonomia em missões prolongadas.  

Por fim, a criação de uma interface de monitorização remota, acessível via aplicação web ou móvel, permitirá acompanhar em tempo real o estado do \gls{usv}, planear rotas e consultar o histórico das missões.  

A realização de ensaios em condições ambientais adversas será também fundamental para avaliar a resiliência, a fiabilidade e a maturidade do sistema em cenários operacionais de maior exigência.  

Em síntese, a evolução do sistema deverá concentrar-se no aumento da autonomia, da fiabilidade e da aplicabilidade prática, consolidando o seu papel enquanto plataforma experimental e operacional para investigação e desenvolvimento na área dos veículos de superfície não tripulados \gls{usv}.
