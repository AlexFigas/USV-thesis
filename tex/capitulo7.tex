\chapter{Conclusões e Trabalho Futuro}
\label{ch:conclusoesTrabalhoFuturo}

O trabalho desenvolvido permitiu conceber e validar um sistema ciberfísico modular para o controlo autónomo de embarcações do tipo \gls{usv}. O protótipo integrou propulsores \emph{brushless}, sensores de navegação (\gls{gps} e \gls{imu}), comunicação de longo alcance via \gls{lora}, bem como uma arquitetura de \emph{software} estruturada em módulos reutilizáveis. Os resultados obtidos demonstraram a viabilidade da solução em cenários reais, confirmando a robustez, a escalabilidade e o potencial de evolução do sistema.

Conclui-se que a abordagem proposta constitui uma base sólida para futuras investigações e aplicações práticas no domínio da robótica marítima com recurso a veículos de superfície não tripulados. O sistema desenvolvido provou ser capaz de operar em modo manual e automático e seguir rotas definidas por \emph{waypoints} \gls{gps}, representando um contributo relevante para a área.

No que respeita ao trabalho futuro, identificam-se várias direções de evolução. Em primeiro lugar, destaca-se a possibilidade de integração do sistema no \gls{usv}-enautica1 da Escola Superior Náutica Infante D. Henrique, permitindo validar o desempenho em cenários de maior escala. Outra vertente importante consiste na implementação de um sistema de controlo remoto manual, quer através de um comando dedicado semelhante aos utilizados em rádio controlo, quer por meio da integração com o comando desenvolvido em \cite{catamara-telecomandado}, assegurando assim a compatibilidade com plataformas operacionais. 

A expansão para guardar telemetria num cartão SD constitui igualmente um caminho relevante. Do ponto de vista do controlo, será pertinente implementar e avaliar estratégias mais avançadas, nomeadamente controladores \gls{pid}, \gls{lqr}, de controlo ótimo ou até técnicas baseadas em aprendizagem por reforço, de modo a aumentar a precisão e a robustez da navegação.

Outra linha de evolução consiste na exploração da navegação cooperativa com múltiplos \gls{usv}, o que implica a coordenação distribuída entre embarcações através de \gls{lora} ou tecnologias alternativas de comunicação em rede. Neste mesmo sentido, a integração de sensores adicionais, como GNSS multiconstelação, câmaras ou radar de baixo custo, permitirá reforçar a precisão da navegação e a resiliência do sistema em ambientes mais complexos.

A sustentabilidade energética poderá ser reforçada através da integração de painéis solares ou de sistemas híbridos de alimentação, aumentando a autonomia em missões prolongadas. Paralelamente, o desenvolvimento de uma interface de monitorização remota, acessível via aplicação web ou móvel, poderá facilitar o acompanhamento em tempo real e o registo histórico das missões realizadas. Finalmente, a realização de ensaios em condições ambientais adversas permitirá avaliar a resiliência e a fiabilidade do sistema em cenários reais de maior exigência.

Em síntese, a evolução do sistema deverá orientar-se para o aumento da autonomia, da fiabilidade e da aplicabilidade prática, consolidando o seu papel enquanto plataforma experimental e operacional para investigação e desenvolvimento na área dos veículos marítimos não tripulados \gls{usv}.
